\documentclass[a4paper]{article}

\usepackage{amsmath}
\usepackage[margin=2.5cm]{geometry}
\usepackage{listings}

\title{OTP in Rust}
\author{Andreas Nicolaisen - \texttt{jtc313} \\ Marco Aslak Persson - \texttt{bfr555}}
\date{\today}

\begin{document}

\maketitle

\section{Introduction}

\section{Background}
\textit{What's OTP?}

\section{Analysis}
\textit{Will probably have a similar structure to the Suture blog-post}
\begin{itemize}
\item What parts of OTP are we interested in porting.
\item How does this translate to Rust?
\item What does Rust make easier/harder?
\item What different things could an implementation focus on?
  \textit{Like performance, robustness, idiomaticity, expressiveness, ergonomics}.

  We'll try to find out whether it \textit{can} be done idiomaticly and ergonomically.

\item What have we chosen to focus on?
\end{itemize}

\section{Solution}
\begin{itemize}
\item Talk about what an OTP implementation overall consists of (in general)
\item Talk about our high-level design/implementation
\item Talk about some concrete implementation decisions (and what drove them)
\item Go into some (few) low-level technical details that might be interesting
\end{itemize}

\section{Examples}
\begin{itemize}
\item High-level architecture (with nice diagrams \texttt{:)})
\item Motivation, i.e. what does this example demonstrate
\item What does this example show/what can we observe/conclude from it
\end{itemize}

\section{Evaluation}


\section{Conclusion}

\end{document}
